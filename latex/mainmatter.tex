% !TeX encoding = UTF-8
% !TeX root = MAIN.tex

\chapter{Motivation}

The starting point of this thesis is recently conducted research that studies the link and possible causal effects between monetary policy decisions by the FED and the stock market in the U.S., not only in an ex-post but also in an ex-ante sense.

\cite{cieslak_stock_2019} (2019) finds a pattern in financial markets around the world that suggests that stock market excess returns in the last 23 were entirely earned in even weeks (0, 2, 4 and 6) starting from the last FOMC meeting. The authors tie their findings to a known phenomenon called "FED Put", by which they mean accommodating monetary policy.

\cite{cieslak_economics_2021} (2021) use textual analysis of FOMC (Federal Open Market Commitee) scripts to identify that policymakers indeed pay attention to the stock market, especially since the mid-1990s and stock market performance is linked with updates of the FED’s internal growth projections.  The authors further claim that even if the policymakers seem to be aware that a dynamic like the FED Put could cause overly risky investing behavior leading to moral-hazard implications,  it does not particularly change their decision-making in an ex-ante sense.

In my thesis,  I aim to find out whether the financial pattern regarding stock excess returns in FOMC even weeks is still relevant after 2016/from 2016 onwards (which should not be the case, especially once the pattern was known to the market and probably complicated by the COVID-19 crisis) since the paper published in 2019 only investigates this pattern before 2016. Additionally, the authors measure the relevance of the financial pattern worldwide using exchange-traded funds (ETFs) containing European stocks. I want to include further results with a specific focus on European stock returns.

%In my thesis I aim to find out whether the "FED Put" remains relevant in the %Eurozone from 2016 onwards till 2023  and whether a similar dynamic can %be observed within the European Central Bank's policy, particularly concerning interest rates and Exchange-Traded Funds (ETFs) containing %European stocks. 

\chapter{What is the "FED Put" and how can it be explained?}
\section{The FED Put}
The FED Put in general refers to (or, moreover, to the belief of) a strong accommodating monetary policy by the Federal Reserve (FED), by which, in case of a sharp decline in asset prices, the FED is expected by the market (its investors) to intervene. The term is coined from the concept of a "Put option" in asset markets, which gives the buyer the option to sell at a predetermined price. Thus, the FED would protect an investor from the decline in the value of an asset.\footnote{\url{https://corporatefinanceinstitute. com/resources/economics/FED-Put/}}

Central banks have gained credibility ever since the mid-1980s by keeping inflation low. \parencite{hall_is_2011} The related term "Greenspan Put" is often used to describe the monetary policy of the under former Federal Reserve Chairman Alan Greenspan to intervene in financial markets in order to prevent significant declines or disruptions. 

While some argue that market interventions are necessary to prevent financial crises (like the Dot-com bubble burst in 2001 or Lehman Brothers in 2008), others believe that these interventions distort the market and create unnecessary moral hazards \parencite{cieslak_economics_2021},  meaning that investors are willing to take on excessive risks because they believe that the FED will always come to rescue them arguing that such "too big to fail” beliefs or mentality can lead to financial crises in the long run.

\section{Stock Returns over the FOMC Cycle}

Diving further into dynamics like the FED Put,  the paper “Stock Returns over the FOMC cycle” focusses on a FOMC cycle specific pattern of the equity premium since 1994. 
The stock returns exhibit a distinct,  statistically significant patterns over the FOMC cycle.  Notably,  it primarily accrues in even weeks of FOMC cycle time (for a graphical explanation of the FOMC cycle, see figure~\ref{cies19_fig2} on page~\pageref{cies19_fig2}). For calculation of stock excess returns they authors use research portfolio data provided by Kenneth R. French for convenience\footnote{\url{https://mba.tuck.dartmouth.edu/pages/faculty/ken.french/data_library.html\#research}}.

\begin{figure}[h]
    \centering
    \label{cies19_fig1}
    \includegraphics[width=0.9\textwidth]{figures/cies19/fig1}
    \caption{Average 5-day stock excess returns over FOMC cycle time (pct) \parencite{cieslak_stock_2019} }
\end{figure}

The authors present three different trading strategies (A, B, and C) that demonstrate on the influence of the FOMC cycle on stock market returns.  Of particular note is Strategy A, which involves exclusively holding stocks during even FOMC cycle weeks. This strategy demonstrates that the average annual returns more than double compared to holding for example an ETF throughout the entire FOMC cycle.  Conversly, the authors find that holding an ETF during uneven FOMC weeks resultet in financial losses over the examined period from 1994 to 2016.  This results have also been covered in the media.\footnote{\url{https://www.economist.com/finance-and-economics/2016/09/03/the-long-arm-of-the-FED}}

\begin{figure}[h]
    \centering
     \label{FED_long_arm}
    \includegraphics[width=0.6\textwidth]{figures/20160903_FNC453.png}
    \caption{\cite{noauthor_long_2016}}
\end{figure}

The authors extend their analysis to explore whether the FOMC cycle return pattern extends beyond the United States, potentially influenced by movements of the dollar currency. To investigate this, they use ETFs containing globally diversified stocks. To establish causality, the authors compare FOMC cycles with other macroeconomic news calendars (e.g., Bloomberg macroeconomic news), dispelling the notion that macroeconomic news significantly correlates with FOMC cycle calendars. They also provide evidence that the release of quarterly firm profits does not substantially account for the observed equity premium patterns over the FOMC cycle.

The authors examine a causal link between the FED's policy actions and the behavior of the stock market by analyzing changes in the Federal Fund target changes between meetings,  FED funds futures and internal meetings of the Board of Governors. They propose that the FED's accommodating policies have a substantial impact on the stock market, resulting in a decrease of the overall equity premium.  Additionally, they contend that there is evidence of informal communication channels between FED officials, the media, and the financial sector, which serve as a means for disseminating information about monetary policy to the market.

\begin{figure}[h]
    \centering
    \label{cies19_fig3A}
    \includegraphics[width=0.9\textwidth]{figures/cies19/fig3A}
    \caption{Probability of Federal Funds Rate (FFR) target changes within FOMC cycle time \parencite{cieslak_stock_2019} }
\end{figure}

\pagebreak

\section{The Economics of the FED Put}
"The Economics of FED Put" \parencite{cieslak_economics_2021} further attempts to study the economics of the relationship between FED policy and the stock market. The authors compare the stock market's predictive power to other economic indicators to forecast changes in the Federal Funds Rate (FFR) using textual analysis from former Federal Open Market Committee (FOMC) meeting transcripts.\footnote{\url{https://www.federalreserve.gov/monetarypolicy/fomc_historical.htm}} Their findings affirm that the FED indeed pays a lot of attention to the stock market during market downturns.

\begin{figure}[h]
    \centering
        \label{cies21_fig5}
    \includegraphics[width=0.9\textwidth]{figures/cies21/Figure5}
    \caption{Negative and positive phrases of the stock market count \parencite{cieslak_economics_2021}}
\end{figure}

They argue that the FED Put is fueled by the Federal Reserve's concerns about the consumption wealth effect. Conversely, strong stock market performance corresponds to updates of the FED’s internal growth projections. Empirical evidence substantiates their claims, as multiple regressions on changes in the Federal Funds Rate (FFR) demonstrate that the stock market captures a higher proportion of the variance (R-squared) compared to other macroeconomic indicators. Significantly, this relationship appears to be less pronounced before the 1990s period.  

During the third European Central Bank (ECB) annual research conference\footcite{european_central_bank_third_2018} in the year of 2018, valuable comments on the econometric approach by the authors were made by the discussant, Emmanuel Moench, the former head of research at Deutsche Bank. Moench suggests that the correlation between negative stock excess returns and the Federal Funds Rate is heavily influenced by two specific FOMC meetings (during financial crises like the dot-com bubble burst in 2001 and the 2008 financial crisis). Furthermore, he recommended incorporating additional covariates, including consumer confidence news and credit spreads, into the regression models to enhance their explanatory power. Moench sees the stock market as one of several co-factors influencing Federal Reserve policy (presumably over the updates of the FED's growth projections as stated by the authors), rather than a dominant driver of the FED's policy.



\chapter{Stock returns over the FOMC Cycle Revisited }


\section{The FOMC cycle}

The FOMC (Federal Open Market Committee) meets approximately every eight weeks during the year,  resulting in an FOMC cycle time of approximately 7 weeks (excluding weekends) most of the time since a year has 52 weeks. The authors, therefore, define FOMC cycle time week dummy variables for week 0 as days -1 to 3, week 1 as days 4 to 8, and week 6 as days 29 to 33. Worth mentioning is that the authors drop 3 days, which would be in FOMC cycle week 7 from their investigation, and that the number of available data points decreases for FOMC dummies (meaning 920 days in week 0,  924 days in week 2,  831 days in week 4,  120 days in week 6 for the relevant timespan from 1994 to 2016).

\label{cies19_fig2}
\begin{figure}[h]
    \centering
    \includegraphics[width=0.75\textwidth]{figures/cies19/fig2}
    \caption{Frequency of FOMC meetings during the year from 1994 to 2016 \parencite{cieslak_stock_2019}}
\end{figure}


\section{FOMC data}

The FOMC meeting dates get published ...


\section{Measurement and estimation analysis (MEA)}

\subsection{MLR model}

One relevant multiple linear regression (MLR) model with dummy variables for even FOMC cycle weeks as in \parencite{cieslak_stock_2019} can be defined as:
\begin{equation}
	rxpct_{i}=\hat{\beta_{0}}+D_0*\hat{\gamma_{1}}+D_1*\hat{\gamma_{2}}+\epsilon_i
\end{equation}
where
$ { \hat{\beta_{0}} } $ is the OLS-estimated intercept,
$ { \hat{\gamma_{1}}, \hat{\gamma_{2}} } $ the OLS-estimated parameters,
${ rx_{i} } $ the excess returns as calculated like in chapter 3.3.4., 
\begin{equation}
    D_0=
    \begin{cases}
      1, & \text{if in the 0 week within FOMC cycle time. }\\
      0, & \text{otherwise}
    \end{cases}
\end{equation}
the FOMC cycle dummy for week 0, 
\begin{equation}
    D_1=
    \begin{cases}
      1, & \text{if in the 2,4 or 6 week within FOMC cycle time. } \\
      0, & \text{otherwise}
    \end{cases}
\end{equation}
the FOMC cycle dummy for week 2, 4, 6 and
$ { \epsilon_i \; \sim \; i.i.d.  \; \mathcal{N}\left(0, \sigma^2 \right) } $
are independent identically distributed OLS-estimated standard errors. 

\subsection{FOMC dummies}

The R Code in \texttt{generate\_fomc\_dummies\_cycle\_dummies.R} (see R Code Appendix)
generates FOMC week dummy variables for later estimation of the influence on FOMC meeting dates on excess stock returns.

\subsection{Data Preprocessing}

The analysis commences with the importation and organization of two distinct datasets. 
The first dataset, identified as \texttt{fomc\_data}, is loaded from the file \texttt{fomc\_week\_dummies\_1994\_nov2023.csv}. 
This dataset encompasses information related to FOMC week dummies spanning from November 1994 to November 2023. The data is sorted by date, and the sorted dataset is then saved as \texttt{d:fomc\_data}, thereby replacing any pre-existing file.

Following this, the second dataset, labeled as \texttt{us\_returns\_data}, is imported from the file \texttt{us\_returns\_df\_1994\_oct2023.csv}. This dataset contains information regarding Fama-French factors for the U.S. market, covering the period from October 1994 to October 2023. Similar to the first dataset, it undergoes sorting by date, and the sorted dataset is saved as \texttt{d:us\_returns\_data}, replacing any existing file.

To consolidate the information, a merge operation is executed using the "date" variable as the key. This operation combines the \texttt{fomc\_data} and \texttt{us\_returns\_data} datasets into a new dataset named \texttt{fed\_put\_datamerged\_data}. The merged dataset is saved as \texttt{d:fed\_put\_datamerged\_data}, effectively replacing any prior file.

Finally, a new variable named \texttt{date2} is generated by transforming the existing "date" variable into Stata date format. This conversion is carried out using the \texttt{date()} function with the "YMD" (year-month-day) format. The resulting dataset is now prepared for further analysis, incorporating information from both the FOMC week dummies and U.S. market returns datasets.

\subsection{Calculation of stock excess returns}

Excess stock returns are calculated using the Fama-French 3-factor model developed by Kenneth R. French and Eugene Fama.  
Data for US market returns for this model and also for various other markets (e.g., European, Asia) get published regularly on Kenneth R. French's webpage. \parencite{kenneth_r_kenneth_nodate}

If \(m\) represents \(1 + \text{{stock return}}\) and \(r\) denote \(1 + \text{{bill return}}\), the 1-day excess return (\text{{ex1}}) is calculated by subtracting \(r\) from \(m\) and multiplying the result by 100, which can be expressed as \(\text{{ex1}} = 100 \times (m - r)\). The 5-day excess return (\text{{ex5}}) is computed over a rolling 5-day window, involving the product of five consecutive values of \(m\) and \(r\). The formula is given by \(\text{{ex5}} = 100 \times (m \times m_{t+1} \times m_{t+2} \times m_{t+3} \times m_{t+4} - r \times r_{t+1} \times r_{t+2} \times r_{t+3} \times r_{t+4})\).

Furthermore, \(t\) represents the observation number in the dataset. 
Overall, the calculation for evaluating stock excess returns provides insight into their 
performance relative to the risk-free rate.

\subsection{Results of the MEA}

\begin{table}[h]
\begin{center}
\begin{adjustbox}{width=1\textwidth}

{
\def\sym#1{\ifmmode^{#1}\else\(^{#1}\)\fi}
\begin{tabular}{l*{3}{c}}
\hline\hline
            &\multicolumn{1}{c}{(1)}   &\multicolumn{1}{c}{(2)}   &\multicolumn{1}{c}{(3)}   \\
            &   2014-2016   &   1994-2014   &   1994-2016   \\
\hline
Dummy = 1 in Week 0    &       0.174*  &       0.138***&       0.143***\\
            &      (1.92)   &      (2.80)   &      (3.21)   \\
[1em]
Dummy = 1 in Week 2,  4,  6     &       0.166** &      0.0890** &      0.0990***\\
            &      (2.55)   &      (2.38)   &      (2.95)   \\
[1em]
Intercept      &     -0.0486   &     -0.0164   &     -0.0206   \\
            &     (-1.14)   &     (-0.76)   &     (-1.05)   \\
\hline
Observations        &         782   &        5224   &        6006   \\
significant at 1\%-level (***), 5\% level (**), 10\% level (*)

\end{tabular}
}

\end{adjustbox}
\caption{\label{table_1} Replication results of Table 1 Panel A as in \parencite{cieslak_stock_2019}}
\end{center}
\end{table}


% \subsection{Stock returns over the FOMC cycle from 2016 onwards}


\begin{table}[h]
\begin{center}
\begin{adjustbox}{width=1\textwidth}
{
\def\sym#1{\ifmmode^{#1}\else\(^{#1}\)\fi}
\begin{tabular}{l*{4}{c}}
\hline\hline
            &\multicolumn{1}{c}{(1)}   &\multicolumn{1}{c}{(2)}   &\multicolumn{1}{c}{(3)}   &\multicolumn{1}{c}{(4)}   \\
            &   2016-2019   &   2019-2022   &   2016-2023   &   1994-2023   \\
\hline
Dummy = 1 in Week 0       &      -0.211** &     -0.0952   &      -0.125   &      0.0800** \\
            &     (-2.29)   &     (-0.57)   &     (-1.40)   &      (2.01)   \\
[1em]
Dummy = 1 in Week 2,  4,  6    &     -0.0487   &      0.0578   &      0.0256   &      0.0828***\\
            &     (-0.74)   &      (0.48)   &      (0.41)   &      (2.81)   \\
[1em]
Intercept      &      0.0960** &      0.0108   &      0.0434   &    -0.00622   \\
            &      (2.48)   &      (0.12)   &      (0.94)   &     (-0.34)   \\
\hline
Observations           &         762   &         779   &        1752   &        7772   \\

\end{tabular}
}
\end{adjustbox}
\caption{\label{table_2} Stock Returns over the FOMC Cycle from 2016 onwards}
\end{center}
\end{table}
\

% Answer Q.: Does the stylized fact of stock excess returns are mainly achieved in FOMC even weeks (0,  2,  4,  6) from 2016 onwards still persist?

In the first sample from 2016 onwards (2016-2019), where the coefficient for the term for FOMC cycle week 0 is statistically significant on the 5\%-level, the sign of the coefficient turned negative, which has been labeled by the media as a "Fed Call." \parencite{noauthor_fed_nodate}

Looking at the whole period from 1994 to 2023, the regression coefficient of the FOMC cycle pattern turns out to be significantly smaller.

All samples from COVID-19 onwards seem to be statistically insignificant so far, suggesting that the FOMC cycle pattern has probably decreased or vanished.


%\subsection{European Stock Returns over the FOMC Cycle from 2016 onwards }

% \subsection{ Stock returns over the FOMC cycle from 2016 onwards European Stock Returns}



%

\chapter{Implications For The Euro-Zone And European Stock Markets}
%Is there empirical evidence for a similar effect when considering only the euro-zone and euro-zone stock returns.  Does it imply an equivalent of the Fed Put in the Euro-Zone?

\newpage

\section{Relevance of the FOMC cycle pattern using European ETFs}

Empirical part II

Substitute for euro-zone only stock excess returns as regressor variable.


Answer Q: Is there empirical evidence for a similar effect when considering only the euro-zone and euro-zone stock returns. 


\newpage


\section{ECB policy meetings and ECB key interest rates}

\newpage

\section{Euro-zone implications}

Empirical part III

Regress on ~"ECB policy meeting dates"

Answer Q: Does it imply an equivalent of the Fed Put in the Euro-Zone?


-Can anything be learned from regressions on past excess returns?
-R-squared,  missing controls/covariates (consumer confidence news, credit spreads ,etc.
-causality?
-How much do 2 specific events excluded from the results change the regression results/variation?


\newpage

\chapter{Conclusion}
In conclusion, the examination of stock excess returns during Federal Open Market Committee (FOMC) even weeks (0, 2, 4, 6) from 2016 onwards show patterns that are distinct from previous sample periods, also when comparing US and European stock markets. The comparison on the coefficients on the dummy variables in Table \ref{table:table_5} display the differences in the responses of the two markets during the FOMC cycle.  

In the period before COVID-19 (2016-2019),  US stocks exhibited a negative "FED Call" with a coefficient of -0.211, statistically significant at the 5\%-level in FOMC week 0, while European stocks showed a less pronounced negative response with a statistically insignificant (indifferent from 0 for all significance-levels) coefficient of -0.106. Similarly, during FOMC week 2, 4, 6, US stocks display a statistically insignificant coefficient of -0.0487,  contrasting the statistically insignificant positive coefficient of 0.00678 for European stocks. 

Focussing on the entire revisited sample period (1994-2023) results in statistically significantly smaller regression coefficients for the FOMC cycle pattern.  Overall, samples from COVID-19 onwards exhibit a lack of statistical significance,  suggesting a potential decrease or disappearance of the FOMC cycle pattern.




