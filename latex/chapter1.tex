\chapter{What is the "Fed Put" and how can it be explained?}
\section{The FED Put}
The "Fed Put" in general refers to (or, moreover, to the belief of) a strong accommodating monetary policy by the Federal Reserve (FED), by which, in case of a sharp decline in asset prices, the FED is expected by the market (its investors) to intervene. The term is coined from the concept of a "put option" in asset markets, which gives the holder the right to sell at a predetermined price. Thus, the Fed would protect an investor from the decline in the value of an asset.  \parencite{noauthor_fed_nodate-1}

Central banks have gained credibility ever since the mid-1980s by keeping inflation low. The related term "Greenspan Put" is often used to describe the monetary policy of the Federal Reserve under the leadership of former Chairman Alan Greenspan to intervene in financial markets in order to prevent significant declines or disruptions.  \parencite{hall_is_2011}

While some argue that market interventions are necessary to prevent financial crises (like the Dot-com bubble burst in 2001 or Lehman Brothers in 2008), others believe that these interventions distort the market and create unnecessary moral hazards \parencite{cieslak_economics_2021},  meaning that investors are willing to take on excessive risks because they believe that the Federal Reserve will always come to rescue them. Some believe that such “too big to fail” beliefs can lead to financial crises in the long run.

\section{Stock Returns over the FOMC Cycle}

Diving further into dynamics like the Fed put,  the paper “Stock returns over the FOMC cycle” focusses on a FOMC cycle specific pattern of the equity premium since 1994. 
The stock returns exhibit a distinct,  statistically significant patterns over the FOMC cycle.  Notably,  it primarily accrues in weeks 0,  2,  4,  and 6 within the FOMC cycle weeks (for a graphical explanation of the FOMC cycle, see figure~\ref{cies19_fig2} on page~\pageref{cies19_fig2}).

For calculation of stock excess returns they authors use research portfolio data provided by Kenneth R. French for convenience.  \parencite{kenneth_r_kenneth_nodate}

\label{cies19_fig1}
\begin{figure}[h]
    \centering
    \includegraphics[width=0.9\textwidth]{figures/cies19/fig1}
    \caption{Average 5-day stock returns minus T-bill returns over the FOMC cycle changes in percent \parencite{cieslak_stock_2019} }
\end{figure}

The authors present three distinct trading strategies (A, B, and C) that shed light on the influence of the FOMC cycle on stock market returns. Of particular note is Strategy A, which involves exclusively holding stocks during even FOMC cycle weeks. This strategy demonstrates that the average annual returns more than double compared to holding an ETF throughout the entire FOMC cycle. Intriguingly, the authors find that holding an ETF during uneven FOMC weeks results in financial losses over the examined period from 1994 to 2016.

The authors extend their analysis to explore whether the FOMC cycle return pattern extends beyond the United States, potentially influenced by movements of the dollar currency. To investigate this, they use ETFs containing globally diversified stocks. To establish causality, the authors compare FOMC cycles with other macroeconomic news calendars (e.g., Bloomberg macroeconomic news), dispelling the notion that macroeconomic news significantly correlates with FOMC cycle calendars. They also provide evidence that the release of quarterly firm profits does not substantially account for the observed equity premium patterns over the FOMC cycle.

To establish a causal link between the Fed's policy measures and stock market behavior, the authors study intermeeting target changes, Fed funds futures, and internal Board of Governors meetings. The authors suggest a significant influence by the Fed over the stock market through its accommodating policies, leading to reductions in the equity premium. Moreover, they argue to uncover evidence of systematic informal communication channels between Fed officials and the media and financial sector, serving as a channel through which news about monetary policy has reached the market.

\label{cies19_fig3A}
\begin{figure}[h]
    \centering
    \includegraphics[width=0.9\textwidth]{figures/cies19/fig3A}
    \caption{Probability of Federal Funds rate changes over the FOMC Cycle \parencite{cieslak_stock_2019} }
\end{figure}

\pagebreak

\section{The Economics of the FED Put}
"The Economics of FED Put" further attempts to study the economics of the relationship between Fed policy and the stock market. The authors compare the stock market's predictive power to other economic indicators to forecast changes in the Federal Funds Rate (FFR) using textual analysis from former Federal Open Market Committee (FOMC) meeting transcripts. Their findings affirm that the Fed indeed pays a lot of attention to the stock market during market downturns.

%\begin{figure}[h]
%    \centering
%    \includegraphics[width=0.9\textwidth]{figures/cies21/Figure5}
%    \caption{Negative and positive phrases of the stock market count 
%\parencite{cieslak_economics_2021}}
%\end{figure}

They argue that the Fed put is fueled by the Federal Reserve's concerns about the consumption wealth effect. Conversely, strong stock market performance corresponds to updates of the Fed’s internal growth projections. Empirical evidence substantiates their claims, as multiple regressions on changes in the Federal Funds Rate (FFR) demonstrate that the stock market captures a higher proportion of the variance (R-squared) compared to other macroeconomic indicators. Significantly, this relationship appears to be less pronounced before the 1990s period.  \parencite{cieslak_economics_2021}

During the third European Central Bank (ECB) research conference, valuable comments on the econometric approach used by the authors were made by the discussant, Emmanuel Moench, the former head of research at Deutsche Bank. Moench suggests that the correlation between negative stock excess returns and the Federal Funds Rate is heavily influenced by two specific FOMC meetings (during financial crises like the dot-com bubble burst in 2001 and the 2008 financial crisis). Furthermore, he recommended incorporating additional covariates, including consumer confidence news and credit spreads, into the regression models to enhance their explanatory power. Moench sees the stock market as one of several co-factors influencing Federal Reserve policy (presumably over the updates of the Fed's growth projections as stated by the authors), rather than a dominant driver of the Fed's policy \parencite{european_central_bank_third_2018}

